%!TEX output_directory = aux
%!TEX aux_directory = aux

\documentclass[10pt]{beamer}

\usetheme[
  numbering=fraction,
  block=fill,
  % background=dark
  ]{metropolis}
%\setbeamercovered{transparent}

\usepackage{appendixnumberbeamer}
\usepackage{booktabs}
\usepackage[scale=2]{ccicons}
\usepackage{pgfplots}

% ------------------------ %
% 	  Packages		%
% ------------------------ %

\usepackage[utf8]{inputenc}
\usepackage[T1]{fontenc}
\usepackage{xcolor}
\usepackage[setbb,setcolon]{kmath}
\usepackage{xparse,pgffor,ifthen}
\usepackage{amsmath,amssymb,amsfonts,amsthm}
\usepackage{mathrsfs,mathtools,bbm}
\usepackage{booktabs,dcolumn,multirow,stmaryrd,array}
\newcolumntype{d}{D{.}{.}{3.2}}
\newcolumntype{e}{D{.}{.}{1.2}}
\usepackage{pgfplots,forest}
\pgfplotsset{compat=newest}
\usepackage[ruled,linesnumbered]{algorithm2e}
\usepackage{cleveref,csquotes,comment,lipsum,todonotes,soul,extarrows,graphicx,xspace}
\usepackage{glossaries}
\usepackage[absolute,overlay]{textpos}
\usepackage{adjustbox}

% \newtheorem{myproof}{Proof}

\usepackage[numbers,sort&compress]{natbib}
\usepackage{glossaries,url,subcaption}
\include{headings/shortcuts}
% ------------------------ %
% 			Macros		   %
% ------------------------ %

% Fonts
\newcommand{\fontset}[1]{\mathcal{#1}}
\newcommand{\fontpb}[1]{#1}
\newcommand{\fontfunc}[1]{\mathrm{#1}}

% Data
\newcommand{\pdim}{n}
\newcommand{\ddim}{m}
\newcommand{\obs}{\y}
\newcommand{\dic}{\A}
\newcommand{\atom}{\a}
\newcommand{\reg}{\lambda}
\newcommand{\lossfunc}{f}
\newcommand{\regfunc}{g}
\newcommand{\lipschitz}{L}
\newcommand{\enetparam}{\sigma}

% Screening sets
\newcommand{\setnodesymb}{\mathcal{S}}
\newcommand{\setzero}{\setnodesymb_0}
\newcommand{\setone}{\setnodesymb_1}
\newcommand{\setnone}{\bar{\setnodesymb}}
\newcommand{\subzero}[1]{#1_{\setzero}}
\newcommand{\subone}[1]{#1_{\setone}}
\newcommand{\subnone}[1]{#1_{\setnone}}

% Problems
\newcommand{\ppb}{\fontpb{P}}
\newcommand{\dpb}{\fontpb{D}}
\newcommand{\pobj}{p}
\newcommand{\dobj}{d}
\newcommand{\pfunc}{\fontfunc{P}}
\newcommand{\dfunc}{\fontfunc{D}}
		
% Variables
\newcommand{\pvi}[1]{x_{#1}}
\newcommand{\dvi}[1]{u_{#1}}
\newcommand{\pv}{\mathbf{x}}
\newcommand{\dv}{\mathbf{u}}

% Index
\newcommand{\idxentry}{i}
\newcommand{\idxscr}{i}
\newcommand{\idxrlx}{i}

% Convex screening
\newcommand{\safesphere}{\mathbb{S}}
\newcommand{\spherecenter}{\mathbf{c}}
\newcommand{\sphereradius}{r}
\newcommand{\matred}{\mathbf{B}}
\newcommand{\vecred}{\mathbf{b}}
\newcommand{\matnorm}{\mathbf{M}}
\newcommand{\reduced}[1]{\tilde{#1}}

% Graphics 
\newcommand{\pointsize}{0.025}
\newcommand{\aPG}{\texttt{aPG}\xspace}
\newcommand{\aPGs}{\texttt{aPG+S}\xspace}
\newcommand{\aPGr}{\texttt{aPG+R}\xspace}
\newcommand{\aPGsr}{\texttt{aPG+S\&R}\xspace}

% Operators
\newcommand{\abs}[1]{|#1|}
\newcommand{\argmin}{\mathrm{argmin}}
\newcommand{\argmax}{\mathrm{argmax}}
\newcommand{\conj}[1]{#1^*}
\newcommand{\card}[1]{|#1|}
\newcommand{\grad}{\nabla}
\newcommand{\Icvx}{\mathbb{I}}
\newcommand{\intervint}[2]{\llbracket#1,#2\rrbracket}
\newcommand{\norm}[2]{\|#1\|_#2}
\newcommand{\opt}[1]{#1^{\star}}
\newcommand{\scalprod}[2]{\langle{#1,#2}\rangle}
\newcommand{\sign}[1]{\mathrm{sign}({#1})}
\newcommand{\subdiff}{\partial}
\newcommand{\relu}[1]{[#1]_+}
\newcommand{\transp}[1]{#1^{T}}

\DeclarePairedDelimiter\ceil{\lceil}{\rceil}
\DeclarePairedDelimiter\floor{\lfloor}{\rfloor}

\include{headings/glossaries}

% \setbeamercovered{transparent} 

\title{Screen \& Relax}
\subtitle{Accélérer la résolution de l'Elastic-Net par identification du support de la solution}
\date{\textbf{GRETSI -- 8 Septembre 2022}}
\author{Théo Guyard${}^{1,2}$, Cédric Herzet${}^{2}$, Clément Elvira${}^{3}$ \\ \scriptsize{${}^{1}$ INSA Rennes} \\ \scriptsize{${}^{2}$ INRIA Rennes Bretagne Atlantique} \\ \scriptsize{${}^{3}$ IETR CentraleSupélec}}

\begin{document}

\begin{frame}
  \maketitle
\end{frame}

\section{General context}

\begin{frame}{Sparse problem}
  \textbf{Ingredients of the problem}
  \begin{itemize}
    \item A \emphone{target} $\obs$
    \pause
    \item A \emphone{dictionary} $\dic=\{\atom_{\idxentry}\}_{\idxentry\in\mathcal{I}}$ made of \emphone{atoms} 
  \end{itemize}
  \pause
  \textbf{Objective}
  \begin{itemize}
    \item Find a \emphone{sparse} linear combination of atoms that \emphone{well approximates} the target through a given model
  \end{itemize}
  \pause
  \textbf{Rough formulation}
  \begin{center}
    \begin{minipage}{0.6\linewidth}
      \begin{block}{Problem}
        \centering
        Find $\pv$ \emphone{sparse} such that \emphone{$\obs \simeq$ Model$(\dic \pv)$}
      \end{block}
    \end{minipage}
  \end{center}
  \pause
  Remark : Entries of $\pv$ weight each atom in the linear combination.
\end{frame}

\section{The Elastic-Net problem}

\begin{frame}{Formulation and properties}
  \begin{block}{Target problem}
    Solve
    \begin{equation}
      \label{prob:primal}
      \tag{$\ppb$}
      \opt{\pv} = \textstyle\arg\min_{\pv} 
      \Big\{ 
        \pfunc(\pv) = 
        \underbrace{
          \tfrac{1}{2}\norm{\obs-\dic\pv}{2}^2
        }_{
          \emphone{\lossfunc(\dic\pv)}
        }
      +
        \underbrace{
          \reg
          (\enetparam\norm{\pv}{1} + \tfrac{1-\enetparam}{2}\norm{\pv}{2})
        }_{
          \emphone{\reg\regfunc(\pv)}
        }
      \Big\}
    \end{equation}
    where $\reg > 0$ and $\enetparam \in ]0,1[$ are tuning hyperparameters.
  \end{block}
  
  \pause
  \textbf{Properties of \eqref{prob:primal}}
  \begin{itemize}
    \item Convex non-smooth problem
    \item Least-squares : Ensures a \emphone{good reconstruction} of the target
    \item $\ell_1$-norm : Enforces \emphone{sparsity}
    \item $\ell_2$-norm : Promotes desirable properties
  \end{itemize}

  \pause
  \textbf{Solving \eqref{prob:primal}}
  \begin{itemize}
    \item Broad class of solution methods (gradient-based, pivot-based, ...)
    \item Acceleration strategies (backtracking, \emphone{screening tests}, ...)
  \end{itemize}
\end{frame}

\section{Screening and Relaxing tests}

\begin{frame}{Main idea}
  \textbf{In the context of a sparse problem}
  \begin{itemize}
    \pause \item Where are \emphone{zero} and \emphone{non-zero} entries in $\opt{\pv}$ ?
    \pause \item Can we \emphone{accelerate} a given solution method if we knew some ?
    \pause \item[$\rightarrow$] Spoiler alert : Yes and yes ! We can leverage \emphone{duality}.
  \end{itemize}
  
  \pause
  ~\\
  \begin{minipage}{0.45\linewidth}
    \begin{block}{\emphone{Primal} problem}
      \begin{equation}
        \tag{$\ppb$}
        \opt{\pv} = \textstyle\arg\min_{\pv} 
        \pfunc(\pv)
      \end{equation}
    \end{block}
  \end{minipage}
  \hfill
  \begin{minipage}{0.04\linewidth}
    \begin{center}
      \hspace{-0.25cm}
      {\LARGE$\equiv$}
    \end{center}
  \end{minipage}
  \hfill
  \begin{minipage}{0.47\linewidth}
    \begin{block}{\emphone{Dual} problem}
      \begin{equation}
        \tag{$\dpb$}
        \opt{\dv} = \textstyle\arg\max_{\dv} 
        \dfunc(\dv)
      \end{equation}
    \end{block}
  \end{minipage}
  \begin{center}
    with \emphone{optimality conditions} linking $\opt{\pv}$ and $\opt{\dv}$
  \end{center}
\end{frame}

\begin{frame}{Optimality conditions}

  \begin{minipage}{0.49\linewidth}
    \vspace{0.5cm}
    \begin{center}
      $\qquad\quad\transp{\atom_i}\opt{\dv} \in \reg\subdiff\regfunc(\opt{\pvi{i}})$
    \end{center}
    \pause
    \vspace{-1.5cm}
    \begin{figure}[!ht]
      \small
      \begin{tikzpicture}[
          domain=-1:1,
          xscale = 2.5,
          yscale = 2.5,
          lbl/.style = {font=\scriptsize, fill=white, 
          inner sep=2pt}
      ]
        \node (origin) at (0,0) {};
        \draw[->] (-1,0) -- (1, 0);
        \draw[->] (0,0) -- (0, 0.5);
        %
        \draw[blue,thick,domain=-1:1] plot (\x, {0.25 * abs(\x) + 0.25 * (\x)^2});
        \node[blue] at (1,0.85) {};
        \node (point3) at (-0.5,0.1875) {};
        \fill[red] (point3) circle (\pointsize);
        \node[below left,red] at (point3) {\scriptsize$\subdiff\regfunc(\pvi{\idxentry})<-\enetparam$};
        \node (point1) at (0,0) {};
        \fill[red] (point1) circle (\pointsize);
        \node[below,red] at (point1) {\scriptsize$\subdiff\regfunc(\pvi{\idxentry})=[-\enetparam,\enetparam]$};
        \node (point2) at (0.5,0.1875) {};
        \fill[red] (point2) circle (\pointsize);
        \node[below right,red] at (point2) {\scriptsize$\subdiff\regfunc(\pvi{\idxentry})>\enetparam$};
      \end{tikzpicture}
    \end{figure}
  \end{minipage}
  \pause
  \hfill
  \begin{minipage}{0.40\linewidth}
    \begin{equation*}
      \begin{array}{rcl}
        \abs{\transp{\atom_{\idxentry}}\opt{\dv}} \leq \reg\enetparam &\iff& \opt{\pvi{\idxentry}} = 0 \\
        \abs{\transp{\atom_{\idxentry}}\opt{\dv}} > \reg\enetparam &\iff& \opt{\pvi{\idxentry}} \neq 0 
      \end{array}
    \end{equation*}
  \end{minipage}
  ~\\
  \vspace*{1cm}
  \pause
  \begin{minipage}{0.4\linewidth}
    \begin{tikzpicture}
    \coordinate (A) at (0,0);
    \coordinate (B) at (0,3);
    \coordinate (C) at (3,3);
    \coordinate (D) at (3,0);
    \coordinate (E) at (2.5,1.1);
    \coordinate (F) at (1.2,2.5);

    \path[fill=magenta!15] (A) to (E) -- (2.5,0) to (A);
    \path[fill=magenta!15] (A) to (F) -- (0,2.5) to (A);
    \path[fill=teal!15] (A) to (F) -- (E) to (A);

    \draw (A) -- (E);
    \draw (A) -- (F);
    \draw[thick,->] (A) -- (B);
    \draw[thick,->] (A) -- (D);

    \draw[thick, magenta, ->] (A) -- (5:2) node[right] {$\atom_1$};
    \draw[thick, magenta, ->] (A) -- (16:2) node[right] {$\atom_2$};
    \draw[thick, teal, ->] (A) -- (32:2) node[right] {$\atom_3$};
    \draw[thick, teal, ->] (A) -- (42:2) node[right] {$\atom_4$};
    \draw[thick, teal, ->] (A) -- (60:2) node[right] {$\atom_5$};
    \draw[thick, magenta, ->] (A) -- (85:2) node[right] {$\atom_6$};
    \draw[circle] (2.2, 2.2) node {$\bullet \ \opt{\dv}$};

\end{tikzpicture}
  \end{minipage}
  \pause
  \hfill
  \begin{minipage}{0.59\linewidth}
    We can identify zeros and non-zeros in $\opt{\pv}$ \pause \emphone{... but we need $\opt{\dv}$ !}
  \end{minipage}
\end{frame}

\begin{frame}{Relaxed optimality condition}
  Let \emphone{$\opt{\dv} \in \safesphere(\spherecenter,\sphereradius)$}, then
  \begin{equation*}
    \begin{array}{rcl}
      \abs{\transp{\atom_{\idxentry}}\spherecenter} + \sphereradius \leq \reg\enetparam &\implies& \opt{\pvi{\idxentry}} = 0 \qquad \text{\emphone{(screening test)}} \\
      \pause
      \abs{\transp{\atom_{\idxentry}}\spherecenter} - \sphereradius > \reg\enetparam &\implies& \opt{\pvi{\idxentry}} \neq 0 \qquad \text{\emphone{(relaxing test)}} 
    \end{array}
  \end{equation*}
  \pause
  $\rightarrow$ No $\opt{\dv}$ needed anymore, but only a \emphone{``safe region''} containing it !
  \pause
  \vspace*{1cm}
  \begin{center}
    \begin{tikzpicture}
    \coordinate (A) at (0,0);
    \coordinate (B) at (0,3);
    \coordinate (C) at (3,3);
    \coordinate (D) at (3,0);
    \coordinate (E) at (2.2,0.9);
    \coordinate (F) at (1.6,1.8);
    \coordinate (G) at (2.4,0.4);
    \coordinate (H) at (0.8,2.4);

    \path[fill=magenta!15] (A) to (G) -- (2.5,0) to (A);
    \path[fill=magenta!15] (A) to (H) -- (0,2.5) to (A);
    \path[fill=teal!15] (A) to (F) -- (E) to (A);

    \draw[thick,->] (A) -- (B);
    \draw[thick,->] (A) -- (D);
    \draw (A) -- (E);
    \draw (A) -- (F);
    \draw (A) -- (G);
    \draw (A) -- (H);

    \draw[thick, magenta, ->] (A) -- (5:2) node[right] {$\atom_1$};
    \draw[thick, ->] (A) -- (16:2) node[right] {$\atom_2$};
    \draw[thick, teal, ->] (A) -- (32:2) node[right] {$\atom_3$};
    \draw[thick, teal, ->] (A) -- (42:2) node[right] {$\atom_4$};
    \draw[thick, ->] (A) -- (60:2) node[right] {$\atom_5$};
    \draw[thick, magenta, ->] (A) -- (85:2) node[right] {$\atom_6$};

    \filldraw[color=gray!60, fill=gray!5, very thick](2.9,2.2) circle (0.5);
    \node at (3.6,2.2) {$\safesphere$};
    \draw[circle] (2.9, 2.2) node {$\bullet \ \opt{\dv}$};

\end{tikzpicture}
  \end{center}
\end{frame}

\section{Dimensionality reduction}

\begin{frame}{Leveraging test results}
  \begin{block}{With \emphone{screening} test}
    \emphone{Zero} entries  identified in $\opt{\pv}$ can be \emphone{discarded} from the problem without changing the objective value.
  \end{block}
  \vspace{0.5cm}
  \pause

  \begin{block}{With \emphone{relaxing} test}
    \emphone{Non-zero} entries identified in $\opt{\pv}$ can be expressed as a \emphone{linear combination} of all the other entries.
  \end{block}
\end{frame}

\begin{frame}{Problem reformulation}
  \textbf{Notations}
  $(\setzero,\setone,\setnone)$ : Set of \emphone{zero}/\emphone{non-zero}/\emphone{unclassified} entries in $\opt{\pv}$
  
  \pause
  \begin{center}
    \textbf{Initial problem}
  \end{center}
  \begin{equation*}
    \opt{\pv} = \argmin_{\emphone{\pv \in \kR^{\pdim}}} \ 
    \Big\{
    \pfunc(\pv) = 
      \tfrac{1}{2}\norm{\obs-\dic\pv}{2}^2
      + \reg (
        \enetparam\norm{\pv}{1}
        + \tfrac{1-\enetparam}{2} \norm{\pv}{2}^2
      )
    \Big\}
  \end{equation*}
  \begin{center}
    $\pdim$ dimensional problem
  \end{center}

  \pause
  \vspace{-0.2cm}
  \begin{center}
    \vspace{0.1cm}
    \rotatebox[origin=c]{-90}{\LARGE$\leadsto$}
  \end{center}
  \vspace{-0.2cm}

  \begin{center}
    \textbf{Reduced problem}
  \end{center}
  \begin{equation*}
    \left\{
      \begin{array}{rl}
        \opt{\pv}_{\setnone} &= \argmin_{\emphone{\pv \in \kR^{\card{\setnone}}}} \ 
        \Big\{
        \emphone{\reduced{\pfunc}}(\pv) = 
          \tfrac{1}{2}\norm{\emphone{\reduced{\obs}}-\emphone{\reduced{\dic}}\pv}{2}^2
          + \reg (
            \enetparam\norm{\pv}{1}
            + \tfrac{1-\enetparam}{2} \|\pv\|{}_{\emphone{\matnorm}}^2
          )
        \Big\} \\
        \opt{\pv}_{\setone} &= \matred\opt{\pv}_{\setnone} + \vecred \\
        \opt{\pv}_{\setzero} &= \0
      \end{array}
    \right.
  \end{equation*}
  \pause
  \begin{center}
    $\emphone{\pdim-\card{\setzero}-\card{\setone}}$ dimensional problem (similar structure) \\
    Some linear algebra operations (negligible computational cost)
  \end{center}
\end{frame}

\begin{frame}{Dynamic Screen \& Relax principle}
  \begin{figure}
    \centering
    \begin{minipage}{0.8\linewidth}
      \begin{algorithm}[H]
        \small
        \DontPrintSemicolon
        \SetKwInOut{Input}{Input}
        \Input{$\dic$, $\obs$, $\reg$, $\enetparam$, $\pv^{(0)}$}
        
        \BlankLine
        \((\setzero,\setone,\setnone) \leftarrow (\emptyset,\emptyset,\{1,\dots,\pdim\})\)\;
        \BlankLine

        \While{convergence criterion is not met}{
          %
          \pause Update the current iterate \;
          \pause Update $(\setzero,\setone,\setnone)$ with screening and relaxing tests\;
          \pause Reduce the problem\;
          \pause \If{$\setnone = \emptyset$}{
            The solution is available in closed form
          }
        }
        \caption{``Screen \& Relax'' solving procedure}
      \end{algorithm}
    \end{minipage}
  \end{figure}
\end{frame}

\section{Some numerical results}

\begin{frame}{Experimental setup}
  \textbf{Synthetic data generation}
  \begin{itemize}
    \item Generate the dictionary $\dic$ randomly
    \item Generate a k-sparse vector $\pv^{\dagger}$
    \item Set $\obs = \dic\opt{\pv} + \text{noise}$
    \item Solve \eqref{prob:primal} with a tailored method
  \end{itemize}
  \pause
  \textbf{Concurrent methods}
  \begin{itemize}
    \item Accelerated proximal-gradient algorithm 
    \item Same with \emphone{screening} tests
    \item Same with \emphone{relaxing} tests
    \item Same with \emphone{screening and relaxing} tests 
  \end{itemize}
  \pause
  \textbf{Metrics}
  \begin{itemize}
    \item Duality gap : How close is the objective from its optimal value
    \item FLOPs : Number of linear algebra operations performed
  \end{itemize}
\end{frame}

\begin{frame}{Classical convergence scheme}
  \begin{center}
    \begin{tikzpicture}
      \begin{axis}[
        width=0.8\textwidth,
        height=0.5\textwidth,
        xmode=log,
        ymode=log,
        xlabel={Duality gap},
        ylabel={FLOPs},
        x dir=reverse,
        legend pos=north west,
        legend cell align=left,
        legend style={
          fill = white, 
          fill opacity=0.7,
          text opacity = 1,
          font = \scriptsize,
        }
      ]
        \addplot[cyan,ultra thick] table [x=dgap, y=aPG] {dat/convergence.dat};
        \addplot[blue,ultra thick] table [x=dgap, y=aPGs] {dat/convergence.dat};
        \addplot[orange,ultra thick] table [x=dgap, y=aPGr] {dat/convergence.dat};
        \addplot[red,ultra thick] table [x=dgap, y=aPGsr] {dat/convergence.dat};
        \legend{\aPG,\aPGs,\aPGr (contrib.),\aPGsr (contrib.)}
      \end{axis}
    \end{tikzpicture}
  \end{center}
  \vspace{-0.25cm}
  \pause
  \begin{itemize}
    \item Complexity reduction with screening and/or relaxing
    \pause \item Convergence to machine precision at some point
    \pause \item Gains depend on the sparsity in $\opt{\pv}$
    \begin{itemize}
      \item Too few non-zeros : relaxing has little impact
      \item Too many non-zeros : problem updates become binding
    \end{itemize}
  \end{itemize}
\end{frame}

\end{document}
